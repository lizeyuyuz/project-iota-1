\section{Linear modeling} The basic idea of linear models is to find the
relationship between a response variable and one or more explanatory variables
(regressors). For our purposes, we attempt to define a relationship between the
BOLD signals and the working memory tasks. Mathematically, our model is as
follows: $$ y = X \beta + \epsilon, X = \{1, x_1, x_2, ..., x_k\}$$. To perform
hypothesis tests using this model, we assume linearity of the relationship, and
$E [\epsilon | X ] = 0$, $Var [ \epsilon | X ] = \Omega$. We solve for linear
regression using generalized least square: $\hat{\beta} = (X^T \Omega^{-1}
X)^{-1} X^T \Omega^{-1} Y$. After finding estimates for the $\beta$s, we perform
Student t-test for each $\beta_i$, where $H_0: \beta_i = 0, i = \{1, 2, ...,
k\}$.
